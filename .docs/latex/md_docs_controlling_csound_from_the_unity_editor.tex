In order to control \mbox{\hyperlink{namespace_csound}{Csound}} instruments in standalone gameplay, you will need to use the methods described above. However, you can also control \mbox{\hyperlink{namespace_csound}{Csound}} channels using the Unity Editor while you are developing your game\textquotesingle{}s sounds. To do so, you must provide a short $<$\+Cabbage$>$$<$/\+Cabbage$>$ descriptor at the top of your \mbox{\hyperlink{namespace_csound}{Csound}} files describing the channels that are needed.

This simple descriptor section uses a single line of code to describe each channel. Each line starts with the given channel\textquotesingle{}s controller type and is then followed by combination of other identifiers such as channel(), text(), and range(). The following descriptor sets up 3 channel controllers. A slider, a button and a checkbox(toggle).


\begin{DoxyCode}{0}
\DoxyCodeLine{<Cabbage>}
\DoxyCodeLine{form caption("{}SimpleFreq"{})}
\DoxyCodeLine{hslider channel("{}freq1"{}), text("{}Frequency Slider"{}), range(0, 10000, 0)}
\DoxyCodeLine{button channel("{}trigger"{}), text("{}Push me"{})}
\DoxyCodeLine{checkbox channel("{}mute"{})}
\DoxyCodeLine{</Cabbage>}

\end{DoxyCode}


Each control MUST specify a channel. The range() identifier must be used if the controller type is a slider. The text() identifier can be used to display unique text beside a control but it is not required. If it is left out, the channel() name will be used as the control\textquotesingle{}s label. The caption() identifier, used with form, is used to display some simple help text to the user.

See \href{https://cabbageaudio.com/docs/cabbage_syntax/}{\texttt{ Cabbage Widgets}} for more information about the syntax to use. \mbox{\hyperlink{class_csound_unity}{Csound\+Unity}} aims to support most of the (relevant) Widgets available in Cabbage.

When a \mbox{\hyperlink{namespace_csound}{Csound}} file which contains a valid $<$\+Cabbage$>$ section is dragged to a \mbox{\hyperlink{class_csound_unity}{Csound\+Unity}} component, Unity will generate controls for each channel. These controls can be tweaked when your game is running. Each time a control is modified, its value is sent to \mbox{\hyperlink{namespace_csound}{Csound}} from Unity on the associated channel. In this way it works the same as the method above, only we don\textquotesingle{}t have to code anything in order to test our sound. For now, \mbox{\hyperlink{class_csound_unity}{Csound\+Unity}} support only four types of controller, slider, checkbox(toggle), button and comboboxes. 